\documentclass{beamer}
\usepackage[utf8]{inputenc}
\usepackage{amsmath, amssymb}
\usetheme{Madrid}

\title{Non-parametric Estimation of Causal Effects on \\ Spatially Clustered Survival Data}
\author{Your Name}
\date{\today}

%%%%%%%%%%%%%%%%%%%%%%%%%%%%%%%%%%%%%%%%%%%%%%%%%%%%%%%%%%%%%%
\begin{document}

%%%%%%%%%%%%%%%%%%%%%%%%%%%%%%%%%%%%%%%%%%%%%%%%%%%%%%%%%%%%%%
\begin{frame}
  \titlepage
\end{frame}

%%%%%%%%%%%%%%%%%%%%%%%%%%%%%%%%%%%%%%%%%%%%%%%%%%%%%%%%%%%%%%
\begin{frame}
  \frametitle{Outline}
  \tableofcontents
\end{frame}

%%%%%%%%%%%%%%%%%%%%%%%%%%%%%%%%%%%%%%%%%%%%%%%%%%%%%%%%%%%%%%
\section{Introduction \& Motivation}

\begin{frame}
\frametitle{Research Motivation and Context}
\begin{itemize}
    \item \textbf{Objective}: Estimate causal effects on survival outcomes using a non-parametric framework.
    \item \textbf{Challenge}: The survival data is \emph{spatially clustered} meaning observations within the same geographical area tend to share unmeasured features.
    \item \textbf{Why Account for Clustering?} 
    \begin{itemize}
      \item Ignoring spatial associations can result in biased treatment effect estimates.
      \item Proper clustering improves efficiency by accounting for correlated errors.
    \end{itemize}
    \item \textbf{Non-parametric Rationale}: Methods such as BART and its spatial extension (SBART) naturally capture non-linearities and interactions without a rigid model specification.
\end{itemize}
\vspace{0.3cm}
\footnotesize References: \textit{Chipman et al. (2010)}; \textit{Hill (2011)}
\end{frame}

\begin{frame}{Why is Clustering Important in Causal Inference?}
    \begin{itemize}
      \item \textbf{Within-Cluster Homogeneity:} 
        \begin{itemize}
          \item Subjects in the same cluster share unobserved factors (e.g., environmental, institutional).
          \item Ignoring this can lead to confounding and biased estimates.
        \end{itemize}
      \item \textbf{Accurate Standard Error Estimation:}
        \begin{itemize}
          \item Independence assumptions fail within clusters.
          \item Clustering adjustment prevents underestimation of standard errors and overconfident inferences.
        \end{itemize}
      \item \textbf{Mitigation of Omitted Variable Bias:}
        \begin{itemize}
          \item Unobserved cluster-specific confounders can impact both treatment and outcome.
          \item Hierarchical models account for these, reducing bias in causal effect estimates.
        \end{itemize}
    \end{itemize}
    \vspace{0.3cm}
    \footnotesize \textbf{References:} Imbens \& Rubin (2015); Wooldridge (2010); Gelman \& Hill (2007)
  \end{frame}
\begin{frame}
    \frametitle{Importance of Clustering in Causal Analysis}
    \begin{itemize}
        \item \textbf{Unmeasured Commonalities:}  
              Grouped individuals (e.g., same hospital or neighborhood) share latent traits that impact outcomes.
              \begin{itemize}
                  \item *Reference:* Wooldridge (2010); Imbens and Rubin (2015)
              \end{itemize}
        \item \textbf{Statistical Implications:}  
              Ignoring clusters can lead to biased estimates and underestimated standard errors, which adversely affects inference.
    \end{itemize}
    \end{frame}
    

    \begin{frame}{Why is Spatial Clustering Important in Causal Inference?}
        \begin{itemize}
          \item \textbf{Shared Environmental Exposures:}
            \begin{itemize}
              \item Nearby units often experience similar environmental, socioeconomic, or healthcare conditions.
              \item Ignoring spatial proximity may lead to unmeasured confounding.
            \end{itemize}
          \item \textbf{Spatial Autocorrelation:}
            \begin{itemize}
              \item Outcomes for spatially close observations are correlated.
              \item Correctly modeling spatial autocorrelation prevents biased estimates and improves uncertainty quantification.
            \end{itemize}
          \item \textbf{Enhanced Precision:}
            \begin{itemize}
              \item Accounting for spatial effects increases model efficiency and robustness.
              \item Hierarchical spatial models capture latent geographical factors impacting treatment effects.
            \end{itemize}
        \end{itemize}
        \vspace{0.3cm}
        \footnotesize \textbf{References:} Anselin (1995); Tobler's First Law of Geography; Linero (2020)
      \end{frame}
    \begin{frame}
        
    \frametitle{Importance of Spatial Dependencies in Causal Inference}
    \begin{itemize}
        \item \textbf{Environmental & Socioeconomic Influences:}  
              Geographic proximity implies similar exposures (e.g., air quality, access to care) that can confound treatment effects.
              \begin{itemize}
                  \item *Reference:* Anselin (1995); Tobler's First Law of Geography.
              \end{itemize}
        \item \textbf{Spatial Random Effects:}  
              Incorporating a spatial component (e.g., via a Gaussian Process) helps account for unobserved spatial heterogeneity, thus improving the validity of causal estimates.
              \begin{itemize}
                  \item *Reference:* Linero (2020)
              \end{itemize}
    \end{itemize}
    \end{frame}
%%%%%%%%%%%%%%%%%%%%%%%%%%%%%%%%%%%%%%%%%%%%%%%%%%%%%%%%%%%%%%
\section{Why Use BART/SBART?}

\begin{frame}
\frametitle{Advantages of Non-parametric Methods for Causal Inference}
\begin{itemize}
    \item \textbf{Flexibility}: Sum-of-trees models adapt to complex relationships between treatment, covariates, and outcomes.
    \item \textbf{Bayesian Framework}: Direct uncertainty quantification and regularization help mitigate overfitting.
    \item \textbf{Spatial Extensions}: SBART augments the standard BART model by incorporating spatial random effects, crucial when data exhibit geographical clustering.
    \item \textbf{Empirical Justification}: Studies show that non-parametric methods outperform classical parametric approaches when dealing with heterogeneous effects and complex interactions.
\end{itemize}
\vspace{0.2cm}
\footnotesize References: \textit{Chipman et al. (2010)}, \textit{Linero (2020)}
\end{frame}


\begin{frame}{Why Use BART/SBART for Causal Inference?}
    \begin{itemize}
      \item \textbf{Flexibility:}  
        \begin{itemize}
           \item BART models the regression function as a sum-of-trees, capturing nonlinearities and complex interactions without a fixed parametric form.
           \item Uncertainty is quantified via posterior draws, which is essential for reliable causal inference.
        \end{itemize}
      \item \textbf{Handling Spatial Clustering:}  
        \begin{itemize}
           \item SBART extends BART by incorporating spatial random effects (e.g., using a CAR prior) to adjust for correlated outcomes within clusters.
           \item This reduces bias and improves the precision of treatment effect estimates in clustered data settings.
        \end{itemize}
      \item \textbf{Empirical Evidence:}  
        \begin{itemize}
           \item \textit{Chipman et al. (2010)} and \textit{Hill (2011)} demonstrate the benefits of BART in causal contexts.
           \item \textit{Linero (2020)} shows that accounting for spatial correlations enhances causal effect estimation.
        \end{itemize}
    \end{itemize}
  \end{frame}
%%%%%%%%%%%%%%%%%%%%%%%%%%%%%%%%%%%%%%%%%%%%%%%%%%%%%%%%%%%%%%


\subsection{Method}
\begin{frame}{Model for Propensity Score}
   
  $P[Z_{ij}=1\mid \mathbf{X}_{ij},\mathbf{V_i}]=e_{ij}, \quad logit(e_{ij})=g(\mathbf{X}_{ij}),$ where $g(\cdot)\sim BART$
  
  
  
  \vspace{30pt}
  \textbf{Two-stage implementation:} (i) estimate PS $\hat{e}_{ij}$, (ii) plug in $\hat{e}_{ij}$ into the survival model. (Doubly Robust)
  
  \end{frame}


\begin{frame}{The AFT-BART Model with Spatial CAR Prior}
  \begin{itemize}
    \item \textbf{Model:} 
      \[
      \log T_{ij} = f\Bigl(Z_{ij},\mathbf{X}_{ij},\mathbf{V}_i,\hat{e}(\mathbf{X}_{ij},\mathbf{V}_i)\Bigr) + W_i + \epsilon_{ij},\quad \epsilon_{ij}\sim N(0,\sigma^2).
      \]
    \item \textbf{Spatial Random Effects:} 
      \[
      p(W\mid \tau^2,\rho) \propto \exp\Bigl\{-\frac{1}{2\tau^2}\,W^\top (D-\rho A)W\Bigr\},
      \]
      where \(A\) is the spatial adjacency matrix, \(D\) is diagonal with \(d_{ii}=\sum_{i'}A_{ii'}\), and \(\rho\) is the spatial parameter.
    \item \textbf{BART:} 
      \[
      f(Z_{ij},\mathbf{X}_{ij},\mathbf{V}_i,\hat{e}(\mathbf{X}_{ij},\mathbf{V}_i))=\sum_{h=1}^{H} g(Z_{ij},\mathbf{X}_{ij},\mathbf{V}_i,\hat{e}(\mathbf{X}_{ij},\mathbf{V}_i);\mathcal{T}_h,\mathcal{M}_h).
      \]
  \end{itemize}
\end{frame}

\begin{frame}{Prior Specification}
  \begin{itemize}
    \item Error variance: \( \sigma^2 \sim \mathrm{IG}(a_\sigma,b_\sigma) \).
    \item Spatial variance: \( \sigma_W^2 \sim \mathrm{IG}(a_W,b_W) \).
    \item Spatial correlations: \(\rho \sim Uniform(\frac{1}{\alpha_{(1)}},\frac{1}{\alpha_{(K)}})\). \(\alpha_{(1)},\alpha_{(K)}\) are the minimum and maximum eigenvalues of \(A\), respectively.
    \item Function \( f(\cdot) \sim \operatorname{SBART}\)
  \end{itemize}
\end{frame}

\begin{frame}{Observed Data Likelihood}
  \begin{itemize}
    \item For each subject \(j\) in cluster \(i\), we observe
      \[
      y_{ij} = \min(T_{ij}, C_{ij}), \quad \delta_{ij} = \mathbbm{1}(T_{ij} < C_{ij}).
      \]
    \item Define the model mean (on the log-scale) as
      \[
      \mu_{ij} = f(Z_{ij}, \mathbf{X}_{ij}, \mathbf{V}_i, \hat{e}(\mathbf{X}_{ij},\mathbf{V}_i)) + W_i.
      \]
    \item Then the individual likelihood contribution is
      \[
      L_{ij}^{\text{obs}}(\theta) = \left\{ \frac{1}{\sqrt{2\pi\sigma^2}}
      \exp\!\left[-\frac{(\log y_{ij} - \mu_{ij})^2}{2\sigma^2}\right] \right\}^{\delta_{ij}}
      \left\{ 1 - \Phi\!\left(\frac{\log y_{ij} - \mu_{ij}}{\sigma}\right) \right\}^{1-\delta_{ij}},
      \]
      where \(\Phi(\cdot)\) is the standard normal CDF.
    \item The full observed-data likelihood is
      \[
      L_{\text{obs}}(\theta) = \prod_{i=1}^K \prod_{j=1}^{n_i} L_{ij}^{\text{obs}}(\theta).
      \]
  \end{itemize}
\end{frame}

\begin{frame}{ Data Augmentation}
  \begin{itemize}
    %\item \textbf{Priors:} Regularizing priors are assigned to the BART tree structures \(\{\mathcal{T}_h\}\) and terminal node parameters \(\{\mu_{lh}\}\).
    \item  Define the latent log survival time \( \tilde{y}_{ij}\) as
      \[
      \tilde{y}_{ij} =
      \begin{cases}
        \operatorname{TruncNormal}\Bigl(\mu_{ij},\sigma^2;\log y_{ij}\Bigr), & \text{if } \delta_{ij}=0, \\[1ex]
        \log y_{ij}, & \text{if } \delta_{ij}=1.
      \end{cases}
      \]
      Here, \(\operatorname{TruncNormal}(\mu,\sigma^2;a)\) denotes a \(N(\mu,\sigma^2)\) distribution truncated to the interval \((a,\infty)\). The imputed values are used in the complete-data likelihood.
  \end{itemize}
\end{frame}

\begin{frame}{Complete Data Likelihood}
  \begin{itemize}
    \item Introduce the latent (complete) log survival times:
      \[
      \tilde{y}_{ij} =
      \begin{cases}
        \log y_{ij}, & \delta_{ij}=1, \\[1ex]
        \text{draw from } N(\mu_{ij},\sigma^2) \text{ truncated to } [\log y_{ij},\infty), & \delta_{ij}=0.
      \end{cases}
      \]
    \item With \(\mu_{ij}\) defined as before, the complete-data likelihood is
      \[
      L_{\text{complete}}(\theta) = \prod_{i=1}^K \prod_{j=1}^{n_i} \frac{1}{\sqrt{2\pi\sigma^2}}
      \exp\!\left[-\frac{(\tilde{y}_{ij} - \mu_{ij})^2}{2\sigma^2}\right].
      \]
  \end{itemize}
\end{frame}


% Slide: Algorithm 1: A Single Iteration
\begin{frame}{Algorithm 1: A Single Iteration}
  \begin{enumerate}
    \item \textbf{Update Spatial Random Effects \& Variance:} Update \(W\), \(\tau^2\), and \(\rho\) from their full conditionals based on the CAR prior.
    
    \item \textbf{Impute Censored Data:} For subjects $ij$, sample the latent log survival time as
      \[
      \tilde{y}_{ij} =
      \begin{cases}
        \operatorname{TruncNormal}\Bigl(\mu_{ij},\sigma^2;\log y_{ij}\Bigr), & \text{if } \Delta_{ij}=0, \\[1ex]
        \log y_{ij}, & \text{if } \Delta_{ij}=1.
      \end{cases}
      \]
    \item \textbf{Update BART:} With responses \(\tilde{y}_{ij} - W_i\) and covariates \((z_{ij},\mathbf{x}_{ij})\), update the BART parameters \(\{\mathcal{T}_h,\mathcal{M}_h\}\) and the error variance \(\sigma^2\) via Bayesian backfitting.
    
  \end{enumerate}
\end{frame}
 



\section{Conclusion \& Future Work}

\begin{frame}
\frametitle{Conclusion and Future Directions}
\begin{itemize}
    \item \textbf{Summary}: We presented a non-parametric approach combining BART with a spatial Gaussian Process component to estimate causal effects in spatially clustered survival data.
    \item \textbf{Benefits}:
    \begin{itemize}
       \item Captures complex, non-linear interactions without requiring pre-specified model forms.
       \item Incorporates spatial dependence to reduce bias and improve uncertainty quantification.
    \end{itemize}
    \item \textbf{Future Work}: 
    \begin{itemize}
       \item Extend the approach to incorporate time-varying covariates.
       \item Enhance computational efficiency for larger datasets.
    \end{itemize}
\end{itemize}
\end{frame}

\begin{frame}
\frametitle{Thank You}
\begin{center}
Questions?
\end{center}
\end{frame}

%%%%%%%%%%%%%%%%%%%%%%%%%%%%%%%%%%%%%%%%%%%%%%%%%%%%%%%%%%%%%%
\begin{frame}[allowframebreaks]
\frametitle{References}
\normalsize
\begin{thebibliography}{99}
\bibitem{chipman2010bart} Chipman, H.A., George, E.I. and McCulloch, R.E., 2010. BART: Bayesian additive regression trees. \emph{The Annals of Applied Statistics}, 4(1), pp.266--298.
\bibitem{hill2011} Hill, J., 2011. Bayesian nonparametric modeling for causal inference. \emph{Journal of Computational and Graphical Statistics}, 20(1), pp.217--240.
\bibitem{linero2020} Linero, A.R., 2020. Bayesian additive regression trees for spatial data. \emph{Journal of the American Statistical Association} (reference details as appropriate).
\end{thebibliography}
\end{frame}

\end{document}